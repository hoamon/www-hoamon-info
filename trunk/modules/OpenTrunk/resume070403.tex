% resume.tex
% 我的履歷
% 使用 bsmi
\documentclass[12pt,a4paper]{article}
\usepackage{CJKutf8}
\usepackage{type1cm}
\usepackage{multirow}
\usepackage[margin=2cm]{geometry}
\newcommand{\minitab}[2][l]{\begin{tabular}{#1}#2\end{tabular}}

\title{Resume}
\author{HO. YUEH FENG}
\date{2007.4}

\parindent=24pt

\begin{document}
\begin{CJK}{UTF8}{bsmi}
    \maketitle
    \fontsize{14pt}{0.8cm}\selectfont
	花了2年的時間才決定報考中興大學營建管理博士班,%
	主要的考量是「我是否適合當一個博士」?我是為了文憑、金錢,還是其他的因素,%
	思忖了很久,也歷經了不少的磨練。到現在,我能堅定說:「唸博士是為了學術研究,%
	是為了求知,是自我的肯定」。我也相信我已經具備從事博士研究的基本能力。	
    \section{自我描述}
    我的個性是不忮不求,脾氣很好。興趣主要是讀書,涉獵範圍廣泛包含財經、%
    管理、數學、物理、電腦作業系統、程式語言、歷史小說等。其他興趣還有打太極拳、%
    網球、桌球及爬山等運動;看棒球、美式足球、電影及推廣開放原始碼。

    在家中排行老大也是獨子,有3個妹妹,這養成我獨立自主負責任的個性。%
    父親是在中華電信工作,母親是標準的家庭主婦,%
    所以讓我的求學生涯能專心於汲取知識上。目前我與家人搬到埔里定居。

    二十七歲時結婚,另一半目前於埔里鎮愛蘭國小擔任老師並兼資訊組長,%
    由於兩人皆熱愛應用資訊技術,所以常常討論網路佈線、%
    網頁伺服器架設及網頁程式設計等問題。

    會走向程式設計/系統管理這條路,%
    要感謝我的研究所指導教授謝孟勳老師\footnote{http://www.ce.nchu.edu.tw/teacher30.htm}%
    在我研二的時候給我一個題目---如何架站!%
    所以我從 ASP + IIS 、 Red Hat Linux 開始玩起,%
    接著是 Apache 、 MySQL 、 PHP 、 Perl ……,逐漸開啟我資訊應用之路。

    對工作的態度是積極作好每一件事情,不會的事物我會主動求知、學習,%
    持續更新個人資訊技術以維持最佳狀態。在團隊中能和同事充分配合,%
    分享資訊、知識,共同完成組識願景。
    \section{學歷}
    \begin{itemize}
        \item 中興大學土木工程學系碩士班營建管理組
        \item 中興大學土木工程學系
        \item 台中二中
        \item 台中市四育國中
        \item 台中市和平國小
    \end{itemize}
    高中時期相當活躍,加入台中二中龍吟社\footnote{辯論社},並擔任公關角色,%
    在班級幹部方面,擔任過副班長、風紀股長。大學時期擔任班級公關、%
    畢冊編輯及「第四屆營建工程與管理研究成果聯合發表會」之事務組組長。%
    研究所時則擔任測量資訊組的總務。

    大學時代,唸的是土木工程學系。中興大學的校風自由、環境清新,%
    也藉此吸收了人文氣息及累積多元能力。且在土木工程學系的四年求學生涯中,%
    反而讓我對於一個大型的營建專案管理有了基本的認識,%
    也因此對我目前所專注的軟體工程\footnote{軟體專案管理。我的興趣主要還是與管理相關。}%
    有莫大的幫助。到了研究所,則選擇了符合興趣的學科-營建管理。%
    畢業論文作的是較偏生產管理-網路規劃應用於多塔吊計劃之最佳化。

    研究所期間並擔任謝孟勳老師的專任助理,處理教學、研究及行政上的一般事務。%
    並在老師兼任推廣教育組組長時,擔任推廣教育中心工讀生,處理在職教育推廣之相關活動。
    \section{經歷}
    當兵期間,選擇文化服務役\footnote{替代役},服務於國史館台灣文獻館,%
    並在地方行政研習中心\footnote{原名:省訓團,廢省後改名。}擔任志工,%
    從事期刊編目及上架的工作。%
    在文獻館的工作是非常多元的。有整理總督府檔案、圖檔整理並掃瞄、%
    後設資料庫系統規劃、委外掃瞄圖檔驗收、出版日治時期市區改正圖錄集、%
    古籍-「風港營雜記」的掃瞄、%
    維護網頁伺服器及開發動態網頁解決難字問題%
    \footnote{http://unicode.hoamon.info/index.plx}等,%
    還有一些運用程式解決問題的雜項工作%
    \footnote{http://hoamon.blogspot.com/2007/01/blog-post.html}。%
    服役期間並獲得九十二年兵役節表揚「績優服勤機關暨獲內政部役政署頒績優團體獎金」%
    等兩項榮譽。%

    第一份工作,任職於中華聯網寬頻股份有限公司,擔任網路工程師,%
    負責無線服務認證系統、公司內部閘道器、檔案伺服器、原始碼版本伺服器、%
    資料庫伺服器、機房建置及員工教育訓練,訓練內容包含系統管理觀念及程式設計技巧等,%
    除了系統管理的工作外,也撰寫動態網頁程式,像是無線服務的信用卡刷卡網頁、%
    無線服務的會員管理系統等。工作內容相當有挑戰性,%
    很多系統都是親手從無至有開發建置而成。%

    第二個工作在暨大擔任行政助理,%
    推動教學卓越計畫子計畫 7-2 \footnote{http://seep.cdc.student.ncnu.edu.tw/}。%
    工作內容有:

    \begin{enumerate}
        \item 畢業生資料匯整分析及撰寫報告。
        \item 規劃企業參訪、讀書會、系友會活動及專案進度管理並撰寫活動成果報告。
        \item 生涯發展中心圖書資訊系統規劃及建置。
        \item 畢業生聯繫平台規劃建置及管理。
        \item 規劃生涯發展中心網站規格。
        \item 各式活動經費核銷。
        \item 其他長官交待事項,如畢業典禮協助…。
    \end{enumerate}

    目前在中興大學營建管理中心擔任技術總監,%
    研究各式資訊新技術應用在營建管理的可行性,像是 AJAX 、 Web Service with Python 、%
    XML Parser 、 Web Framework …等。也在研究 TurboGears 、 django 、%
    Ruby On Rails 、 ASP.NET 、 ActiveRecord 的設計概念,%
    並實作 TWD - Turtle Webpage Dev 專案,讓自己參透網頁程式框架。%
    \section{專長}
    \begin{enumerate}
        \item Data Mining (多變量分析)

        運用多變量分析(主成份分析、集群分析、相關係數分析)來分析龐大無序的資料。
        \item 問題意識及溝通規劃

        對於全新面臨的需求/觀念,能快速有效了解問題及瓶頸所在,%
        並與伙伴分享想法,提出解決方案。
        \item 生產管理(線性規劃)

        主要解決生產組合成本、工率問題、人員分派最適化等。
        \item 營建專案管理/系統分析
        
        營建專案通常具備龐大的特性:時間長、人力雜、物料多。%
        這三個因素也造就了問題多。營建專案的週期分為五段,構想階段、%
        設計階段、發包階段、施工階段、維護階段。愈早階段犯下的錯,%
        付出的修復成本愈高。

        我個人認為營建專案管理可應用於軟體專案管理(軟體工程)上,%
        就像 fred broks 的人月神話\footnote{軟體專案管理名著,至今歷久不衰}%
        一書中所舉的例子,十之八九都是營建工程,像是失敗的巴別塔、%
        莊嚴的大教堂……等。兩者的特性有大半部份是相同的,%
        例如一樣都要寫規格書\footnote{軟體上稱「需求規格書」、「技術規格書」%
        ,在營建工程中,稱「建築圖」、「施工說明書」。}卻都不重視規格書的寫作、%
        業主都不具資訊或營建專長及開發時程常常延誤……。%
        所以從軟體發展的趨勢來看,軟體專案也開始注重外包與整合。
        \item Matlab(高階數學程式語言)

        為實作線性規劃、多變量分析方法的工具。
        \item Open Source應用(開放原始碼應用)

        有效地運用 Open Source 像是 Linux 、 Apache 、 Postfix 、 MySQL 、%
        PostgreSQL 、 BIND 、 GNU RADIUS ……等,能降低成本、提高品質。%
        之前的工作就有好幾個 Open Source 的應用實例:無線服務認證系統、%
        公司內部閘道器、檔案伺服器、文件版本伺服器、資料庫伺服器。%
        現在並計劃自行開發 Open Source 專案來解決無線安全問題、快速開發%
        網頁專案等。
        \item PHP(動態網頁設計)

        寫了幾個小網站。與老婆一起協助幫忙縣政府財產課寫了一個土地管理系統%
        \footnote{http://163.22.168.29/},此專案運用了 FPDF 模組,%
        另外還有自己的網站\footnote{http://www.hoamon.info/}以及 TWD 網頁開發框架。
        \item Perl\footnote{多功能程式語言,乃系統管理員之最強工具}

        用Perl寫了無數個泛 Unix os 的工具程式。Perl是一種很靈活的語言,%
        可以只花費約二十分鐘的時間,就寫好程式來處理資料夾分類、字串抓取、%
        網路抓檔等工作。

        我也用它寫了一個難字資料庫\footnote{http://unicode.hoamon.info/index.plx},%
        負責查找中文難字的 unicode 碼以輸入總督府後設資料庫中。
        \item Python\footnote{Perl的最大競爭者}

        應該還不算專長,目前正在學。但希望不久的將來(約1年吧),%
        這會是我的超級強項。我已把未來壓在這個語言上了。
    \end{enumerate}
    \section{生涯規劃}
    \renewcommand{\multirowsetup}{\centering} 
    \begin{tabular}{|l|c|p{6cm}|p{3cm}|c|} \hline
    時間 & 生活 & 讀書 & 工作 & 旅遊 \\ \hline
    2005 & 遷居埔里 & 充實離散數學、演算法與資料結構及計算機組識與作業系統%
        & 暨大助理 (直到2006-7) & \\ \hline
    2006 & 埔里置產 & Web Framework 、 IPv6 、 SIP 、 Python 相關知識%
        & 興大 CMC 技術總監 (從2006-9開始) & 到蘭嶼 \\ \hline
    2007 & & 報考興大土木系博士班、投稿期刊  & Use Django & 走八通關古道 \\ \hline
    2008 & & 出版 Djagno 書籍 & 移植 PHP 網頁到 Django & \\ \hline
    2009 & & 出版 Python 書籍 &  & 到奧地利旅遊 \\ \hline
    \end{tabular}
    \section{其他}
    \begin{itemize}
        \item 碩士論文: \verb|http://etds.ncl.edu.tw/theabs/site/sh/|

        \verb|search_result.jsp?hot_query=Yueh+Feng+Ho&field=AU|
        \item 個人網站: http://www.hoamon.info/
    \end{itemize}
\end{CJK}
\end{document}
